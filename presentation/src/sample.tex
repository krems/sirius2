\documentclass[russian,aspectratio=169,14pt]{beamer}

\usetheme{ShipilevRH}

\title{Паттерны проектирования}
\subtitle{Приемы написания поддерживаемого кода}
\author{Валерий Алексеевич Овчинников}
\institute{valery.ovchinnikov@phystech.edu}

\begin{document}

\maketitle

\section{Мотивация}

\begin{frame}
	\frametitle{Зачем изучать паттерны?}
	\begin{itemize}
		\item Перенимать опыт
		\pause
		\item Читая чужой код, можно понять, как улучшить свой
		\pause
		\item Заранее узнать о подводных камнях
		\pause
		\item Не изобретать велосипед
	\end{itemize}
\end{frame}

\begin{frame}
	\frametitle{Чего мы хотим?}
	\begin{center}
	\includegraphics[height=0.75\textheight]{want.jpeg}
	\end{center}
\end{frame}

\begin{frame}
	\frametitle{Чего мы хотим?}
	\begin{itemize}
		\item SOLID код
		\item Понятный другим программистам код
		\item Поддерживаемый код
	\end{itemize}
\end{frame}

\begin{frame}
	\frametitle{Как в этом помогают паттерны?}
	\begin{itemize}
		\item Соответствуют принципам SOLID, даже помогают писать SOLID код
		\item Общеизвестны, а значит, понятны другим программистам
		\item Были разработаны для сложных ситуаций (в плане поддержки кода)
	\end{itemize}
\end{frame}

\begin{frame}
	\frametitle{Когда использовать паттерны?}
	\textbf{Только тогда, когда это необходимо}
	\begin{center}
	\includegraphics[height=0.6\textheight]{ko.jpg}
	\end{center}
\end{frame}




\section{Command. Команда}

\begin{frame}
    \begin{center}
	\includegraphics[height=0.8\textheight]{command.jpg}
	\end{center}
\end{frame}

\begin{frame}[fragile]
	\frametitle{Что это?}
	\begin{listjava}
interface Command {
    void execute();
}
class ShoutingCommand implements Command {
    private final String text;
    ShoutingCommand(String text) {
        this.text = text;
    }
    @Override
    public void execute() {
        System.out.println(text.toUpperCase());
    }
}
	\end{listjava}
\end{frame}

\begin{frame}
	\frametitle{Зачем это?}
	\begin{itemize}
		\item Позволяет задавать различное поведение через единый интерфейс
		\pause
		\item Хранит данные и действия над ними рядом
		\pause
		\item Разрывает связь между инициатором операции (знает когда и что) и ее исполнителем (знает как)
		\pause
		\item $^*$Является альтернативой call-back'ам (функциям обратного вызова)
	\end{itemize}
\end{frame}

\begin{frame}
	\frametitle{Примеры использования}
	\begin{itemize}
		\item Executors Framework (Runnable/Callable)
		\item Функторы в C++
		\item Транзакции
		\item В реактивном программировании
	\end{itemize}
\end{frame}




\section{Decorator. Декоратор}

\begin{frame}
    \begin{center}
	\includegraphics[height=0.8\textheight]{decorator.jpg}
	\end{center}
\end{frame}

\begin{frame}[fragile]
	\frametitle{Что это?}
	\begin{listjava}
interface Worker {
    void work();
}
class GardenWorker implements Worker {
    @Override
    public void work() {
        System.out.println("Cutting grass");
    }
}
	\end{listjava}
\end{frame}

\begin{frame}[fragile]
	\frametitle{Что это?}
	\begin{listjava}
class ProfilingWorkerDecorator implements Worker {
    private final Worker worker;
    ProfilingWorker(Worker worker) {
        this.worker = worker;
    }
    @Override
    public void work() {
        long start = System.currentTimeMillis();
        worker.work();
        long duration = System.currentTimeMillis() - start;
        System.out.println("Took " + duration + "ms");
    }
}
	\end{listjava}
\end{frame}

\begin{frame}
	\frametitle{Зачем это?}
	\begin{itemize}
		\item Позволяет расширять функциональность классов не нарушая OCP, LSP
		\pause
		\item Спасает от комбинаторного взрыва
		\pause
		\item Удобен для профилирования (при этом сохраняет SRP)
		\pause
		\item $^*$Позволяет писать в стиле Aspect Oriented Programming
	\end{itemize}
\end{frame}

\begin{frame}
	\frametitle{Примеры использования}
	\begin{itemize}
		\item Input/Output Streams в Java
		\item Добавление графических оформлений виджетам (темы)
		\item Добавление отладочной информации, проверки прав
		\item Proxy класс в Java
		\item Встроенный механизм в язык Python (@)
		\item AspectJ
	\end{itemize}
\end{frame}




\section*{Весёлая задачка}

\begin{frame}
    \frametitle{(на самом деле нет)}
    Прокаченный ExecutorService
    \vfill
    \begin{enumerate}
    	\item Добавить возможность
    	\begin{enumerate}
    		\item Измерять время выполнения задачи
    		\pause
       		\item Ловить все ошибки задачи и выводить их на экран
       		\pause
    		\item Изменять имя потока на имя класса выполняемой задачи
    	\end{enumerate}
    	\pause
    	\item Соответствовать принципам SOLID
    	\pause
    	\item Использовать паттерны Builder, Fabric (method) + Decorator
    	\pause
    	\item Конфигурировать (включать/выключать) новые функции
    \end{enumerate}
\end{frame}




\section{Builder. Строитель}

\begin{frame}
    \begin{center}
	\includegraphics[height=0.8\textheight]{builder.jpg}
	\end{center}
\end{frame}

\begin{frame}[fragile]
	\frametitle{Что это?}
\begin{listjava}
class Person {
	final String firstName;
	final String middleName;
	final String surname;
	final Date dateOfBirth;
	Person(String firstName, String middleName, 
	       String surname, Date dateOfBirth) {
		this.firstName = firstName;
		this.middleName = middleName;
		this.surname = surname;
		this.dateOfBirth = dateOfBirth;
	}
}
\end{listjava}	
\end{frame}

\begin{frame}[fragile]
	\frametitle{Что это?}
\begin{listjava}
class PersonBuilder {
    String firstName;
    ...
    Person build() {
        check();
        return new Person(firstName, middleName, ...);
    }
    
    void withFirstName(String name) {
        this.firstName = name;
    }
    ...
}
\end{listjava}	
\end{frame}

\begin{frame}[fragile]
	\frametitle{Что это?}
\begin{listjava}
PersonBuilder builder = new PersonBuilder();
builder.withFirstName("Oleg");
builder.withMiddleName("Olegovich");
builder.withSurname("Olegov");
builder.bornAt(dateOfBirth);
Person oleg = builder.build();
\end{listjava}	
\end{frame}

\begin{frame}
	\frametitle{Зачем это?}
	\begin{itemize}
		\item Сделать код более читаемым, если в конструкторе много (однотипных) параметров
		\pause
		\item Если нужно создавать неизменяемый объект, но не все его параметры доступны сразу
		\pause
		\item Если нужно делать сложные проверки аргументов
	\end{itemize}
\end{frame}

\begin{frame}
	\frametitle{Примеры использования}
	\begin{itemize}
		\item StringBuilder
		\item Различные парсеры и конструкторы JSON/XML
	\end{itemize}
\end{frame}

\begin{frame}[fragile]
	\frametitle{Fluent interface}
	Часто используется совместно с паттерном Строитель
	\vfill
\begin{listjava}
Person oleg = builder.withFirstName("Oleg")
					.withMiddleName("Olegovich")
					.withSurname("Olegov")
					.bornAt(dateOfBirth)
					.build();
\end{listjava}	
\end{frame}

\begin{frame}[fragile]
	\frametitle{Fluent interface}
\begin{listjava}
class PersonFluentBuilder {
    String firstName;
    ...
    Person build() {
        check();
        return new Person(firstName, middleName, ...);
    }
    
    PersonFluentBuilder withFirstName(String name) {
        this.firstName = name;
        return this;
    }
    ...
}
\end{listjava}	
\end{frame}




\section{Factory. Фабрика}

\begin{frame}
    \begin{center}
	\includegraphics[height=0.8\textheight]{fabric.jpg}
	\end{center}
\end{frame}

\begin{frame}[fragile]
	\frametitle{Что это?}
	Фабрика
	\begin{listjava}
class ElephantFactory {
    static Elephant pinkElephant() {
        return new Elephant("#FFC0CB");
    }
    
    static Elephant africanElephant() {
        return new Elephant(ElephantType.African, "#D3D3D3", 
            getPhysicalParams(ElephantType.African));
    }
}
	\end{listjava}
\end{frame}

\begin{frame}[fragile]
	\frametitle{Что это?}
	Фабричный метод
	\begin{listjava}
public static LocalDate now();
public static LocalDate ofEpochDay(long epochDay);
public static LocalDate ofYearDay(int year, int dayOfYear);
public static LocalDate parse(CharSequence text);
	\end{listjava}
\end{frame}

\begin{frame}
	\frametitle{Зачем это?}
	\begin{itemize}
		\item Сделать код более читаемым, если существует несколько разных конструкторов
		\pause
		\item Скрывает реализации, оставляя открытым только интерфейс
		\pause
		\item Позволяет компоновать семейства из совместимых объектов
		\pause
		\item $^*$Один из способов организовать Inversion of Control
	\end{itemize}
\end{frame}

\begin{frame}
	\frametitle{Примеры использования}
	\begin{itemize}
		\item Executors Framework
		\item LocalDate
		\item Spring ServiceLocator
	\end{itemize}
\end{frame}




\section{Observer. Наблюдатель}

\begin{frame}
    \begin{center}
	\includegraphics[height=0.8\textheight]{observer.jpg}
	\end{center}
\end{frame}

\begin{frame}[fragile]
	\frametitle{Что это?}
	\begin{listjava}
interface Observer<T> {
    void update(T observed);
}
class Publisher {
    Collection<Observer<Publisher>> observers;
    void subscribe(Observer<Publisher> observer) {
        observers.add(observer);
    }
    void publish() {
        for (Observer<Publisher> observer : observers) {
            observer.update(this);
        }
    }
}
	\end{listjava}
\end{frame}

\begin{frame}
	\frametitle{Зачем это?}
	\begin{itemize}
		\item Позволяет оповещать заинтересованные объекты о событиях
		\pause
		\item Механизм обеспечивается через единый интерфейс, избегается жесткая связанность
	\end{itemize}
\end{frame}

\begin{frame}
	\frametitle{Примеры использования}
	\begin{itemize}
		\item В подходе к проектированию Model-View-Controller
		\item В различных обработчиках событий (например, Button.onClick)
		\item В реализации паттерна-подхода Proactor
	\end{itemize}
\end{frame}




\section*{Весёлая задачка}

\begin{frame}
    \frametitle{(на самом деле нет)}
    Прокаченный ExecutorService
    \vfill
    \begin{enumerate}
    	\item Добавить возможность
    	\begin{enumerate}
    		\item Измерять время выполнения задачи
    		\item Ловить все ошибки задачи и выводить их на экран
    		\item Изменять имя потока на имя класса выполняемой задачи
    		\item \textbf{Подписываться на отправку задач в Executor}
    	\end{enumerate}
    	\item Соответствовать принципам SOLID
    	\item Использовать паттерны Builder, Fabric (method) + Decorator
    	\item Конфигурировать (включать/выключать) новые функции
    \end{enumerate}
\end{frame}




\section{Singleton. Уникум/Одиночка. Антипаттерн$^*$}

\begin{frame}
    \begin{center}
	\includegraphics[height=0.8\textheight]{singleton.jpeg}
	\end{center}
\end{frame}

\begin{frame}
	\frametitle{Что это?}
	Объект класса Singleton существует в единственном экземпляре
\end{frame}

\begin{frame}[fragile]
	\frametitle{Что это?}
	\begin{listjava}
class Singleton {
    public static final Singleton INSTANCE = new Singleton();
    
    private Singleton() {}
}
	\end{listjava}
\end{frame}

\begin{frame}[fragile]
	\frametitle{Что это?}
	\begin{listjava}
class Singleton {
    private static Singleton instance;
    
    private Singleton() {}
    
    public static Singleton instance() {
        if (instance == null) {
            instance = new Singleton();
        }
        return instance;
    }
}
	\end{listjava}
\end{frame}

\begin{frame}
	\frametitle{Зачем это?}
	\begin{itemize}
		\item Не дает по ошибке создать несколько инстансов объекта, который всегда должен быть уникальным
		\pause
		\item Часто реализован неправильно, еще чаще не к месту
	\end{itemize}
\end{frame}

\begin{frame}
	\frametitle{Примеры использования}
	\begin{itemize}
		\item Объект, представляющий фаловую систему
		\item Редко -- объект, представляющие соединение с базой
	\end{itemize}
\end{frame}


\questions

\end{document}
